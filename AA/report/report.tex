\documentclass[a4paper,oneside,14pt]{extreport}
\input{common/preamble}


\def\lab{2}
\def\topic{Расстояние Левенштейна и Дамерау-Левенштейна}
\def\topic{Алгоритмы умножения матриц}
% \def\topic{Алгоритмы сортировки}

\begin{document}

\input{common/title}
\tableofcontents

\chapter*{Введение}
\addcontentsline{toc}{chapter}{Введение}


Алгоритм сортировки - это алгоритм для упорядочивания элементов
в списке. В случае, когда элемент списка имеет несколько полей, поле,
служащее критерием порядка, называется ключом сортировки. На практике
в качестве ключа часто выступает число, а в остальных полях хранятся
какие-либо данные, никак не влияющие на работу алгоритма.
\\


\textbf{Целью работы:} изучение алгоритмов сортировки массивов, сравни-
тельный анализ времени работы данных алгоритмов, анализ трудоемкости
алгоритмов.
\\

\textbf{Задачи работы:}

\begin{enumerate}
    \setlength{\itemsep}{0em}
    \item реализовать три различных алгоритма сортировки;
    \item теоретически вычислить эффективность алгоритмов;
    \item сравнить эффективности алгоритмов по времени.
\end{enumerate}

\chapter{Аналитический раздел}
\label{cha:analysis}

В данном разделе будет приведено описание
конвейерной обработки и параллельных вычислений.


\section{Задача коммивояжера}

Задача коммивояжера - одна из самых известных задач комбинаторной оптимизации, заключающаяся в поиске самого выгодного маршрута, проходящего через указанные города хотя бы по одному разу с последующим возвратом в исходный город. В условиях задачи указываются критерий выгодности маршрута (кратчайший, самый дешевый, совокупный критерий и тому подобное) и соответствующие матрицы расстояний, стоимости и тому подобного. Как правило, указывается, что маршрут должен проходить через каждый город только один раз - в таком случае выбор осуществляется среди гамильтоновых циклов.

\section{Полный перебор}

Алгоритм полного перебора для решения задачи коммивояжера
является наиболее прямым решением, он пробует все перестановки (упорядоченные комбинации) и определяет, какой из них самый дешевый.
Этот подход гарантирует точное решение задачи.
Но сложность алгоритма $O(n!)$, факториал количества городов, поэтому это решение становится непрактичным даже для 20 городов.

\section{Муравьиный алгоритм}

Муравьиные алгоритмы представляют собой новый перспективный метод решения задач оптимизации,
в основе которого лежит моделирование поведения колонии муравьев.
Колония представляет собой систему с очень простыми правилами автономного поведения особей.

Моделирование поведения муравьев связано с распределением феромона на
тропе - ребре графа в задаче коммивояжера. При этом вероятность включения
ребра в маршрут отдельного муравья пропорциональна количеству феромона на
этом ребре, а количество откладываемого феромона пропорционально длине
маршрута. Чем короче маршрут, тем больше феромона будет отложено на его
ребрах, следовательно, большее количество муравьев будет включать его в синтез
собственных маршрутов. Моделирование такого подхода, использующего только
положительную обратную связь, приводит к преждевременной сходимости -
большинство муравьев двигается по локально оптимальному маршруту. Избежать
этого можно, моделируя отрицательную обратную связь в виде испарения феромона.
При этом если феромон испаряется быстро, то это приводит к потере памяти колонии и забыванию хороших решений,
с дугой стороны, большое время испарения может привести к получению устойчивого локально оптимального решения.
Теперь, с учетом особенностей задачи коммивояжера, мы можем описать локальные правила поведения муравьев при выборе пути.


\begin{itemize}
	\item Муравьи имеют собственную "<память">. Поскольку каждый город может быть посещен только один раз,
    у каждого муравья сеть список уже посещенных городов — список запретов.
    Обозначим через $J_{i,k}$ список городов, которые необходимо пройти муравью $k$, находящемуся в городе $i$.
	\item Муравьи обладают "<зрением"> -- видимость есть эвристическое желание посетить город $j$, если муравей находится в городе $i$. Будем считать, что видимость обратно пропорциональной расстоянию между соответствующими городами $\eta_{i,j} = 1/D_{i,j}$.
	\item Муравьи обладают "<обонянием"> -- могут улавливать след феромона, подтверждающий желание посетить город $j$ из города $i$ на основании опыта других муравьев. Обозначим количество феромона на ребре $(i,j)$ в момент времени $t$ через $\tau_{i,j}(t)$. 
\end{itemize}

Вероятность перехода из вершины $i$ в вершину $j$ определяется по следующей формуле \ref{form:way}

\begin{equation}\label{form:way} 
	p_{i,j}={\frac {(\tau_{i,j}^{\alpha })(\eta_{i,j}^{\beta })}{\sum (\tau_{i,j}^{\alpha})(\eta_{i,j}^{\beta })}}
\end{equation}

где \quad$ \tau_{i,j} - $ количество феромонов на ребре i до j;

$\eta_{i,j} - $ эвристическое расстояние от i до j;

$\alpha - $ параметр влияния феромона;

$\beta - $ параметр влияния расстояния.


Пройдя ребро $(i,j)$ , муравей откладывает на нем некоторое количество феромона, которое должно быть связано с оптимальностью сделанного выбора. Пусть $T _{k} (t)$ есть маршрут, пройденный муравьем $k$ к моменту времени $t$ , $L _{k} (t)$ - длина этого маршрута, а $Q$ - параметр, имеющий значение порядка длины оптимального пути. Тогда откладываемое количество феромона может быть задано в виде:

\begin{equation}\label{form:add} 
	{\displaystyle \Delta \tau _{i,j}^k={\begin{cases}Q/L_{k}(t),& {(i,j)\in T_k(t)}\\0,&{\mbox{иначе}}\end{cases}}}
\end{equation}

где \quad Q - количество феромона, переносимого муравьем;


Тогда

\begin{equation}\label{form:add1} 
	\Delta \tau _{ij}(t)= \sum_{k=1}^{m} \Delta \tau _{ij,k}(t)
\end{equation}



Сложность данного алгоритма определяется непосредственно из приведенного выше текста --
$O(t_{max} \cdot m \cdot n^2)$, таким образом, сложность зависит от
времени жизни колонии, количества городов и количества муравьев в колонии.


\section{Вывод}
В данном разделе были рассмотрены задача коммивояжера,
алгоритм полного перебора и муравьиный алгоритм для решения данной задачи.

\chapter{Конструкторский раздел}
\label{cha:design}

В данном разделе представлены схемы рассматриваемых алгоритмов.

\section{Разработка алгоритмов}

На рисунках \ref{fig:2.1} и \ref{fig:2.2} приведены схемы алгоритмов решения задачи
коммивояжера перебором и муравьиным алгоритмом соответственно.

\begin{figure}[h!]
    \centering
    \includegraphics[width=0.6\textwidth]{6/inc/d1.png}
    \caption{Схема алгоритма полного перебора решения задачи коммивояжера}
    \label{fig:2.1}
\end{figure}

\begin{figure}[h!]
    \centering
    \includegraphics[width=0.6\textwidth]{6/inc/d2.png}
    \caption{Схема муравьиного алгоритма решения задачи коммивояжера}
    \label{fig:2.2}
\end{figure}



\section{Вывод}
В данном разделе были рассмотрены схемы алгоритмов для решения задачи коммивояжера.
\chapter{Технологический раздел}
\label{cha:impl}

% \section{Требования к программному обеспечению}


\section{Средства реализации}

Язык программирования: C++

Библиотеки: google test, google benchmark

Редактор: VS Code

Я использую эти инструменты потому, что они мощные, широко используемые и
хочу изучить фреймворк для тестирования и тестирования на C ++.



\section{Листинг кода}

Я создал шаблон для матричного типа, сами данные использовал статический двумерный массив.
Его интерфейс прост в использовании, но код не очень понятен. Для краткости я перечисляю только шаблон Matrix.


\lstinputlisting[
    language=c++,linerange={0-7},
    caption=Шаблон для матричного типа
    ]{2/inc/code.h}

\lstinputlisting[
    language=c++,linerange={10-30},
    caption=Стандартный алгоритм умножения матриц
    ]{2/inc/code.h}

\lstinputlisting[
    language=c++,linerange={32-78},
    caption=Алгоритм Винограда для умножения матриц
    ]{2/inc/code.h}

\lstinputlisting[
    language=c++,linerange={80-130},
    caption=Алгоритм Винограда для умножения матриц с оптимизацией
    ]{2/inc/code.h}


\section{Описание тестирования}

В таблице \ref{tabular:func_test} приведен функциональные тесты
для алгоритмов умножения матриц.

\def\arraystretch{1.2}
\setlength\tabcolsep{0.5cm}

\begin{table}[h]
    \centering
    \begin{tabular}{|c|c|c|}
        \hline
        \bfseries Матрица 1  & \bfseries Матрица 2 & \bfseries Ожидаемый результат
        \\\hline
        $\begin{pmatrix} 0 & 0 \\ 0 & 0 \\ \end{pmatrix}$
        &
        $\begin{pmatrix} 2 & 3 \\ 4 & 4 \\ \end{pmatrix}$
        &
        $\begin{pmatrix} 0 & 0 \\ 0 & 0 \\ \end{pmatrix}$
        \\\hline

        $\begin{pmatrix} 2 & 4 & 3 \\ 1 & -3 & 2 \\ \end{pmatrix}$
        &
        $\begin{pmatrix} 2 & -3 \\ 4 & 4 \\ 2 & 3\\ \end{pmatrix}$
        &
        $\begin{pmatrix} 26 & 19 \\ -6 & -9 \\ \end{pmatrix}$
        \\\hline

        $\begin{pmatrix} 2 & -3 \\ 4 & 4 \\ 2 & 3\\ \end{pmatrix}$
        &
        $\begin{pmatrix} 2 & 4 & 3 \\ 1 & -3 & 2 \\ \end{pmatrix}$
        &
        $\begin{pmatrix} 1 & 17 & 0 \\ 12 & 4 & 20 \\ 7 & -1 & 12 \\ \end{pmatrix}$
        \\\hline

        $\begin{pmatrix} 0 & 0 \\ 0 & 0 \\ \end{pmatrix}$
        &
        $\begin{pmatrix} 0 & 0 \end{pmatrix}$
        &
        Exception
        \\\hline
    \end{tabular}
    \caption{\label{tabular:func_test} Функциональные тесты}
\end{table}

\section{Вывод}

В этом разделе было рассмотрено код программы и описание тестирования.
\chapter{Экспериментальный раздел}
\label{cha:research}

В данном разделе будет приведены пример работы программы и сравнение времени работы программы.

\section{Примеры работы}
На рисунке \ref{fig:4.1} приведен пример работы программы.

\begin{figure}[h]
    \centering
    \includegraphics[width=0.8\textwidth]{7/inc/e1.png}
    \caption{Примеры работы программы}
    \label{fig:4.1}
\end{figure}

\section{Результат тестирования}

На рисунке \ref{fig:4.2} приведен результат тестирования.
Словарь состоит из 1000 элементов, были протестированы элементы с номерами 0, 999, 499, 500, 101, 777 и несуществующий ключ.

\begin{figure}[h]
    \centering
    \includegraphics[width=0.6\textwidth]{7/inc/test.png}
    \caption{Результат тестирования}
    \label{fig:4.2}
\end{figure}



% \section{Постановка эксперимента по замеру времени}
% \pagebreak
\section{Сравнение времени работы}

Операционная система - Ubuntu 20.04.1 LTS

Процессор - Intel® CoreTM i5-7300HQ CPU @ 2.50GHz × 4 (ЦП 4 ядра 4 потока)

В таблице \ref{tabular:4.1} приведена таблица частотного анализа.


% \def\arraystretch{1.2}
\setlength\tabcolsep{0.2cm}


\begin{table}[h]
    \centering
    \catcode`\#=12  % deactivate # sign
    \csvreader[tabular=|c|c|,
        table head=\hline
        \bfseries Буква слова
        & \bfseries Количество слов
        \\\hline,
        late after line=\\\hline]
        {7/inc/freq.csv}{}
    { \csvcoli & \csvcolii}
    \catcode`\#=3   % reactivate # sign
    \caption{\label{tabular:4.1} Частотный анализ}
\end{table}


\definecolor{bblue}{HTML}{4F81BD}
\definecolor{rred}{HTML}{C0504D}
\definecolor{ggreen}{HTML}{9BBB59}
\definecolor{ppurple}{HTML}{9F4C7C}

% \clearpage
\begin{figure}[!h]
    \centering
    \begin{tikzpicture}
        \begin{axis}[
            width  = 0.85*\textwidth,
            height = 10cm,
            major x tick style = transparent,
            ybar=2*\pgflinewidth,
            bar width=14pt,
            ymajorgrids = true,
            ylabel = {Время работы, $\mu$с},
            symbolic x coords={Linear,Binary,Hybrid},
            xtick = data,
            scaled y ticks = false,
            enlarge x limits=0.25,
            ymin=0,
            % legend cell align=left,
            % legend style={
            %         at={(1,1.05)},
            %         anchor=south east,
            %         column sep=1ex
            % }
        ]
            \addplot[style={bblue,fill=bblue,mark=none}]
                coordinates {(Linear,6968) (Binary,1336) (Hybrid,920)};
    
            \addplot[style={ggreen,fill=ggreen,mark=none}]
                coordinates {(Linear,107) (Binary,153) (Hybrid,865)};
    
            \addplot[style={rred,fill=rred,mark=none}]
                coordinates {(Linear,12413) (Binary,1628) (Hybrid,1052)};

            \legend{Average,Best,Worst}
        \end{axis}
    \end{tikzpicture}
    \caption{Зависимость времени работы алгоритмов поиска}
    \label{fig:4.4}
\end{figure}

\pagebreak
\section{Вывод}

Эксперимент показывает, что в среднем алгоритм линейного поиска худший, а лучший - гибридный алгоритм (сегментация + бинарный поиск).
Алгоритм линейного поиска не требует сортировки данных, но для отсортированных данных он работает очень быстро, чтобы найти первые (например, найти самую популярную криптовалюту).
Гибридный алгоритм может работать даже лучше с большим количеством сегментов, если вместо линейного поиска сегмента мы будем искать с использованием хэш-карты.
\chapter*{Заключение}
\addcontentsline{toc}{chapter}{Заключение}

В ходе лабораторной работы было изучено основы конвейерной обработки,
реализованна конвейера с использованием параллельных вычислений.
Было сравнить временные характеристики параллельная безконвейерная версия,
конвейер и параллельный конвейер
и сделаны следующие выводы:

\begin{itemize}
    \item параллельный конвейер в 2 раз быстрее, чем конвейер,
    который использует один поток для каждого этапа;
    \item конвейер недостаточно длинный и для его реализации требуются дополнительные каналы и передача данных,
    поэтому не видим здесь разницы между параллельным конвейером и параллельно-безконвейерной реализацией.
\end{itemize}

\addcontentsline{toc}{chapter}{Литература}
\bibliographystyle{ugost2008}


\begin{thebibliography}{8}

    \bibitem{b1}
    Воеводин В. В., Воеводин Вл. В. Параллельные вычисления. — СПб: БХВ-Петербург, 2002. — 608 с.

    \bibitem{b2}
    Конвейерные вычисления
    \\\url{https://studylib.ru/doc/4736512/konvejernye-vychisleniya}
    [Электронный ресурс] (дата обращения: 25.12.20)

    \bibitem{b3}
    Effective Go
    \\\url{https://golang.org/doc/effective_go.html}
    [Электронный ресурс] (дата обращения: 25.12.20)

\end{thebibliography}

\end{document}

% http://detexify.kirelabs.org/classify.html
% https://github.com/poulter7/ipynb-tex


% В данном разделе будет приведено описание схем алгоритмов.
% На рис. 1 представлена схема алгоритма
% определения расстояния Левенштейна в матричной реализации

% В этом разделе анализируются существующие алгоритмы построения трехмерных изображений
% и выбираются наиболее подходящие алгоритмы для решения поставленных задач.

% В данном разделе проектируется новая всячина.

% В данном разделе описано изготовление и требование всячины. Кстати,

% В данном разделе проводятся вычислительные эксперименты.