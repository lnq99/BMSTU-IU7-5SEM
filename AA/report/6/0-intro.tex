\chapter*{Введение}
\addcontentsline{toc}{chapter}{Введение}


Муравьиный алгоритм - один из эффективных полиномиальных алгоритмов для нахождения приближенных решений задачи коммивояжера, а также решения аналогичных задач поиска маршрутов на графах. Суть подхода заключается в анализе и использовании модели поведения муравьев, ищущих пути от колонии к источнику питания, и представляет собой метаэвристическую оптимизацию.
\\


\textbf{Целью работы:}
провести сравнительный анализ метода полного перебора и эвристического метода на базе муравьиного алгоритма.
\\

\textbf{Задачи работы:}

\begin{itemize}
    \setlength{\itemsep}{0em}
    \item реализовать метод полного перебора и метод на базе муравьиного алгоритма для решения задачи коммивояжера с возвращением последнего в город, с которого он начал обход;
    \item провести параметризацию второго метода для выбранного класса задач, т.е. определить такие комбинации параметров или их диапазонов, при которых метод дает наилучшие результаты на выбранном(ых) классе(ах) задач.
\end{itemize}
