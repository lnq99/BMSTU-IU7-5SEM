\chapter{Аналитический раздел}
\label{cha:analysis}

В данном разделе будет приведено описание
конвейерной обработки и параллельных вычислений.


\section{Задача коммивояжера}

Задача коммивояжера - одна из самых известных задач комбинаторной оптимизации, заключающаяся в поиске самого выгодного маршрута, проходящего через указанные города хотя бы по одному разу с последующим возвратом в исходный город. В условиях задачи указываются критерий выгодности маршрута (кратчайший, самый дешевый, совокупный критерий и тому подобное) и соответствующие матрицы расстояний, стоимости и тому подобного. Как правило, указывается, что маршрут должен проходить через каждый город только один раз - в таком случае выбор осуществляется среди гамильтоновых циклов.

\section{Полный перебор}

Алгоритм полного перебора для решения задачи коммивояжера
является наиболее прямым решением, он пробует все перестановки (упорядоченные комбинации) и определяет, какой из них самый дешевый.
Этот подход гарантирует точное решение задачи.
Но сложность алгоритма $O(n!)$, факториал количества городов, поэтому это решение становится непрактичным даже для 20 городов.

\section{Муравьиный алгоритм}

Муравьиные алгоритмы представляют собой новый перспективный метод решения задач оптимизации,
в основе которого лежит моделирование поведения колонии муравьев.
Колония представляет собой систему с очень простыми правилами автономного поведения особей.

Моделирование поведения муравьев связано с распределением феромона на
тропе - ребре графа в задаче коммивояжера. При этом вероятность включения
ребра в маршрут отдельного муравья пропорциональна количеству феромона на
этом ребре, а количество откладываемого феромона пропорционально длине
маршрута. Чем короче маршрут, тем больше феромона будет отложено на его
ребрах, следовательно, большее количество муравьев будет включать его в синтез
собственных маршрутов. Моделирование такого подхода, использующего только
положительную обратную связь, приводит к преждевременной сходимости -
большинство муравьев двигается по локально оптимальному маршруту. Избежать
этого можно, моделируя отрицательную обратную связь в виде испарения феромона.
При этом если феромон испаряется быстро, то это приводит к потере памяти колонии и забыванию хороших решений,
с дугой стороны, большое время испарения может привести к получению устойчивого локально оптимального решения.
Теперь, с учетом особенностей задачи коммивояжера, мы можем описать локальные правила поведения муравьев при выборе пути.


\begin{itemize}
	\item Муравьи имеют собственную "<память">. Поскольку каждый город может быть посещен только один раз,
    у каждого муравья сеть список уже посещенных городов — список запретов.
    Обозначим через $J_{i,k}$ список городов, которые необходимо пройти муравью $k$, находящемуся в городе $i$.
	\item Муравьи обладают "<зрением"> -- видимость есть эвристическое желание посетить город $j$, если муравей находится в городе $i$. Будем считать, что видимость обратно пропорциональной расстоянию между соответствующими городами $\eta_{i,j} = 1/D_{i,j}$.
	\item Муравьи обладают "<обонянием"> -- могут улавливать след феромона, подтверждающий желание посетить город $j$ из города $i$ на основании опыта других муравьев. Обозначим количество феромона на ребре $(i,j)$ в момент времени $t$ через $\tau_{i,j}(t)$. 
\end{itemize}

Вероятность перехода из вершины $i$ в вершину $j$ определяется по следующей формуле \ref{form:way}

\begin{equation}\label{form:way} 
	p_{i,j}={\frac {(\tau_{i,j}^{\alpha })(\eta_{i,j}^{\beta })}{\sum (\tau_{i,j}^{\alpha})(\eta_{i,j}^{\beta })}}
\end{equation}

где \quad$ \tau_{i,j} - $ количество феромонов на ребре i до j;

$\eta_{i,j} - $ эвристическое расстояние от i до j;

$\alpha - $ параметр влияния феромона;

$\beta - $ параметр влияния расстояния.


Пройдя ребро $(i,j)$ , муравей откладывает на нем некоторое количество феромона, которое должно быть связано с оптимальностью сделанного выбора. Пусть $T _{k} (t)$ есть маршрут, пройденный муравьем $k$ к моменту времени $t$ , $L _{k} (t)$ - длина этого маршрута, а $Q$ - параметр, имеющий значение порядка длины оптимального пути. Тогда откладываемое количество феромона может быть задано в виде:

\begin{equation}\label{form:add} 
	{\displaystyle \Delta \tau _{i,j}^k={\begin{cases}Q/L_{k}(t),& {(i,j)\in T_k(t)}\\0,&{\mbox{иначе}}\end{cases}}}
\end{equation}

где \quad Q - количество феромона, переносимого муравьем;


Тогда

\begin{equation}\label{form:add1} 
	\Delta \tau _{ij}(t)= \sum_{k=1}^{m} \Delta \tau _{ij,k}(t)
\end{equation}



Сложность данного алгоритма определяется непосредственно из приведенного выше текста --
$O(t_{max} \cdot m \cdot n^2)$, таким образом, сложность зависит от
времени жизни колонии, количества городов и количества муравьев в колонии.


\section{Вывод}
В данном разделе были рассмотрены задача коммивояжера,
алгоритм полного перебора и муравьиный алгоритм для решения данной задачи.
