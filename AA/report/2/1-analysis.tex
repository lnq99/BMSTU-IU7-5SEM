\chapter{Аналитический раздел}
\label{cha:analysis}

В данном разделе будет приведено описание алгоритмов и модель
вычислений для оценок трудоемкости.

\section{Описание алгоритмов}

\subsection{Стандартный алгоритм}


Пусть даны две прямоугольные матрицы $A$ и $B$ размерности $l$ x $m$ и
$m$ x $n$ соответственно:\\

$
A = \begin{pmatrix}
    a_{11} & a_{12} & \ldots & a_{1m}\\
    a_{21} & a_{22} & \ldots & a_{2m}\\
    \vdots & \vdots & \ddots & \vdots\\
    a_{l1} & a_{l2} & \ldots & a_{lm}
\end{pmatrix},
\qquad
B = \begin{pmatrix}
    b_{11} & b_{12} & \ldots & b_{1n}\\
    b_{21} & b_{22} & \ldots & b_{2n}\\
    \vdots & \vdots & \ddots & \vdots\\
    b_{m1} & b_{m2} & \ldots & b_{mn}
\end{pmatrix}
$\\\\


Тогда матрица $C$ размерностью $l$ x $m$:\\

$
C = \begin{pmatrix}
    c_{11} & c_{12} & \ldots & c_{1n}\\
    c_{21} & c_{22} & \ldots & c_{2n}\\
    \vdots & \vdots & \ddots & \vdots\\
    c_{l1} & c_{l2} & \ldots & c_{ln}
\end{pmatrix}
$\\

в которой:

\begin{equation}
    \label{eq:1.1}
    c_{ij} =
    \sum_{r=1}^{m} a_{ir} b_{rj} \quad (i=1,2,...,l; \ j=1,2,...,n)
\end{equation}

называется их произведением.

\pagebreak
\subsection{Алгоритм Винограда}

Рассматривая результат умножения двух матриц очевидно, что каждый элемент
в нем представляет собой скалярное произведение соответствующих строки и столбца исходных матриц.
Такое умножение допускает предварительную обработку,
позволяющую часть работы выполнить заранее.

Рассмотрим два вектора
$V = (v_1, v_2, v_3, v_4)$ и
$W = (w_1, w_2, w_3, w_4)$.
Их скалярное произведение равно:

\begin{equation}
    \label{eq:1.2}
    V \cdot W = v_1 w_1 + v_2 w_2 + v_3 w_3 + v_4 w_4
\end{equation}


Это равенство можно переписать в виде:

\begin{equation}
    \label{eq:1.2}
    V \cdot W = (v_1+w_2)(v_2+w_1)+(v_3+w_4)(v_4+w_3)-v_1v_2-v_3v_4-w_1w_2-w_3w_4
\end{equation}

Несмотря на то, что второе выражение требует вычисления большего
количества операций, чем первое: вместо четырех умножений - шесть,
а вместо трех сложений - десять, выражение в правой части последнего равенства
допускает предварительную обработку: его части можно вычислить заранее
и запомнить для каждой строки первой матрицы и для каждого столбца
второй, что позволяет выполнять для каждого элемента лишь первые два
умножения и последующие пять сложений, а также дополнительно два сложения.


\subsection{Модель вычислений}

В данной работы используется следующая модель вычислений:

\begin{enumerate}
    \item Стоимость базовых операций: $F = 1$\\
    $(=, *, +, -, /, \%, <, <=, >, >=, ==, !=, [], +=, -=, *=, /=)$

    \item Стоимость цикля $for$\\
    $F_{for} = f_{init} + f_{compare} + N_{loop} \cdot (f_{body} + f_{inc} + f_{compare})$

    \item Трудоемкость условного оператора $if$\\
    $F_{if} = f_{compare} + f_{body} = f_{compare}
    + \left\{
            \begin{array}{ll}
                f_{min}, \hspace{0.5cm} $лучший случай$\\
                f_{max}, \hspace{0.5cm} $худший случай$\\
            \end{array}
            \right.
    $\\
\end{enumerate}

\section{Вывод}
Были приведено описание алгоритмов, стандартный и Винограда,
также рассмотрено модель вычислений для оценок трудоемкости.
