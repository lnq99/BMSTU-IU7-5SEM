\chapter{Технологический раздел}
\label{cha:impl}

% \section{Требования к программному обеспечению}


\section{Средства реализации}

Язык программирования: C++

Библиотеки: google test, google benchmark

Редактор: VS Code

Я использую эти инструменты потому, что они мощные, широко используемые и
хочу изучить фреймворк для тестирования и тестирования на C ++.



\section{Листинг кода}

Я создал шаблон для матричного типа, сами данные использовал статический двумерный массив.
Его интерфейс прост в использовании, но код не очень понятен. Для краткости я перечисляю только шаблон Matrix.


\lstinputlisting[
    language=c++,linerange={0-7},
    caption=Шаблон для матричного типа
    ]{2/inc/code.h}

\lstinputlisting[
    language=c++,linerange={10-30},
    caption=Стандартный алгоритм умножения матриц
    ]{2/inc/code.h}

\lstinputlisting[
    language=c++,linerange={32-78},
    caption=Алгоритм Винограда для умножения матриц
    ]{2/inc/code.h}

\lstinputlisting[
    language=c++,linerange={80-130},
    caption=Алгоритм Винограда для умножения матриц с оптимизацией
    ]{2/inc/code.h}


\section{Описание тестирования}

В таблице \ref{tabular:func_test} приведен функциональные тесты
для алгоритмов умножения матриц.

\def\arraystretch{1.2}
\setlength\tabcolsep{0.5cm}

\begin{table}[h]
    \centering
    \begin{tabular}{|c|c|c|}
        \hline
        \bfseries Матрица 1  & \bfseries Матрица 2 & \bfseries Ожидаемый результат
        \\\hline
        $\begin{pmatrix} 0 & 0 \\ 0 & 0 \\ \end{pmatrix}$
        &
        $\begin{pmatrix} 2 & 3 \\ 4 & 4 \\ \end{pmatrix}$
        &
        $\begin{pmatrix} 0 & 0 \\ 0 & 0 \\ \end{pmatrix}$
        \\\hline

        $\begin{pmatrix} 2 & 4 & 3 \\ 1 & -3 & 2 \\ \end{pmatrix}$
        &
        $\begin{pmatrix} 2 & -3 \\ 4 & 4 \\ 2 & 3\\ \end{pmatrix}$
        &
        $\begin{pmatrix} 26 & 19 \\ -6 & -9 \\ \end{pmatrix}$
        \\\hline

        $\begin{pmatrix} 2 & -3 \\ 4 & 4 \\ 2 & 3\\ \end{pmatrix}$
        &
        $\begin{pmatrix} 2 & 4 & 3 \\ 1 & -3 & 2 \\ \end{pmatrix}$
        &
        $\begin{pmatrix} 1 & 17 & 0 \\ 12 & 4 & 20 \\ 7 & -1 & 12 \\ \end{pmatrix}$
        \\\hline

        $\begin{pmatrix} 0 & 0 \\ 0 & 0 \\ \end{pmatrix}$
        &
        $\begin{pmatrix} 0 & 0 \end{pmatrix}$
        &
        Exception
        \\\hline
    \end{tabular}
    \caption{\label{tabular:func_test} Функциональные тесты}
\end{table}

\section{Вывод}

В этом разделе было рассмотрено код программы и описание тестирования.