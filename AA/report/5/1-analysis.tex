\chapter{Аналитический раздел}
\label{cha:analysis}

В данном разделе будет приведено описание
конвейерной обработки и параллельных вычислений.

\section{Конвейерная обработка}

\textbf{Конвейер} - машина непрерывного транспорта, предназначенная для
перемещения сыпучих, кусковых или штучных грузов.

\textbf{Конвейеризация} - это техника, в результате которой задача или команда разбивается на некоторое числоподзадач, которые  выполняются последовательно.
Каждая подкоманда выполняется на своем логическом устройстве. Все логические устройства (ступени) соединяются последовательно таким образом, что выход i-ой ступени связан свходом (i+1)-ой ступени,все ступени работают одновременно.
Множество ступеней называется конвейером.Выигрыш вовремени достигается привыполнении нескольких задачзасчет параллельной работы ступеней, вовлекая на каждом такте новую задачу или команду.


\section{Оценка производительности конвейера}


Пусть задана операция, выполнение которой разбито на n последовательных этапов. При последовательном их выполнении операция выполняется за время

\begin{equation}
    \label{eq:1.1}
    \tau_{e}={\sum\limits_{i=1}^n \tau_{i}}
\end{equation}
где \\
n - количество последовательных этапов; \\
$\tau_{i}$ - время выполнения i-го этапа;

Быстродействие одного процессора, выполняющего только эту операцию, составит

\begin{equation}\label{eq:1.2}
    S_{e}={\frac{1}{\tau_{e}}}={\frac{1}{\sum\limits_{i=1}^n \tau_{i}}}
\end{equation}
где \\
$\tau_{e}$ - время выполнения одной операции; \\
n - количество последовательных этапов; \\
$\tau_{i}$ - время выполнения i-го этапа;

Выберем время такта - величину $t_{T} = max{\sum\limits_{i=1}^n(\tau_{i})}$ и потребуем при разбиении на этапы, чтобы для любого i = 1, ... , n выполнялось условие $(\tau_{i} + \tau_{i+1}) mod n = \tau_{T}$. То есть чтобы никакие два последовательных этапа (включая конец и новое начало операции) не могли быть выполнены за время одного такта.

Максимальное быстродействие процессора при полной загрузке конвейера составляет

\begin{equation}\label{eq:1.3}
 S_{max}={\frac{1}{\tau_{T}}}
\end{equation}
где \\
$\tau_{T}$ - выбранное нами время такта;

Число n - количество уровней конвейера, или глубина перекрытия, так как каждый такт на конвейере параллельно выполняются n операций. Чем больше число уровней (станций), тем больший выигрыш в быстродействии может быть получен.

Известна оценка

\begin{equation}\label{eq:1.4}
\frac{n}{n/2} \leq \frac{S_{max}}{S_{e}} \leq n
\end{equation}
где \\
$S_{max}$ - максимальное быстродействие процессора  при полной загрузке конвейера; \\
$S_{e}$ - стандартное быстродействие процессора; \\
n - количество этапов.

то есть выигрыш в быстродействии получается от n/2  до n раз.


Реальный выигрыш в быстродействии оказывается всегда меньше, чем указанный выше, поскольку:

\begin{enumerate}
\item некоторые операции, например, над целыми, могут выполняться за меньшее количество этапов, чем другие арифметические операции. Тогда отдельные станции конвейера будут простаивать;
\item при выполнении некоторых операций на определённых этапах могут требоваться результаты более поздних, ещё не выполненных этапов предыдущих операций. Приходится приостанавливать конвейер;
\item поток команд (первая ступень) порождает недостаточное количество операций для полной загрузки конвейера.
\end{enumerate}

\section{Многопоточность}

К достоинствам многопоточной реализации той или иной системы перед многозадачной можно отнести следующее:

\begin{itemize}
    \item Упрощение программы в некоторых случаях за счет использования общего адресного пространства.
    \item Меньшие относительно процесса временные затраты на создание потока.
\end{itemize}

К достоинствам многопоточной реализации той или иной системы перед однопоточной можно отнести следующее:

\begin{itemize}
    \item Упрощение программы в некоторых случаях, за счет вынесения механизмов чередования выполнения различных слабо взаимосвязанных подзадач, требующих одновременного выполнения, в отдельную подсистему многопоточности.
    \item Повышение производительности процесса за счет распараллеливания процессорных вычислений и операций ввода-вывода.
\end{itemize}


Существует два виды параллелизма в алгоритмах и программах:

\begin{itemize}
    \item Конечный параллелизм определяется информационной
    независимостью некоторых фрагментов в тексте программы.
    \item Массовый параллелизм определяется
    информационной независимостью итераций циклов программы.
\end{itemize}

% В этой работе я использую массовый параллелизм.

\section{Вывод}
В данном разделе были приведено описание описание
конвейерной обработки и параллельных вычислений.
